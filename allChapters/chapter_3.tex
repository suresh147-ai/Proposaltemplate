      \chapter{Requirement Analysis}
        \section{SOFTWARE REQUIREMENT}
Our Facial Regonition   system requires Python, OpenCV, MySQL, TensorFlow, Flask,
CNN, JavaScript, Keras which are described below;
            \subsection{Python}
Python is a high-level interpreted programming language for general-purpose
programming created by Guido van Rossum which as released in 1991. Python has
design philosophy that emphasizes code readability, notably using significant white
spaces. It provides constructs that enable clear programming in small and large scales.
The programming language features dynamic type system and automatic memory
management with support to multiple programming paradigms including Object
Oriented Imperative, Functional and Procedural. The programming language has large
comprehensive standard library.
\subsection{OpenCV (Open Source Computer Vision Library)}
OpenCV is a popular open-source library designed for computer vision tasks. It provides a wide range of functions and algorithms related to image and video processing.OpenCV is widely used for tasks like image and video processing, object detection, facial recognition, and more.
It is written in C++ but has Python bindings, making it accessible to a broader audience. 
\subsection{MySQL}
MySQL is an open-source relational database management system (RDBMS). Its name
is a combination of "My", the name of co-founder Michael Widenius's daughter, and
"SQL", the abbreviation for Structured Query Language. In our project, we use this
database to feed into PHP system for data labeling. The database is created from
scraping Nepali Sentiment text in python.
            \subsection{Tensor Flow}
TensorFlow is an open-source deep learning framework developed by the Google Brain team. It is widely used for building and training machine learning and deep learning models.It provides Flexible architecture for building various machine learning models.It supports deep learning, including neural networks and convolutional neural networks (CNNs).TensorFlow is used for developing and training deep learning models for tasks such as image classification, object detection, natural language processing, and more.

   \section{FUNCTIONAL REQUIREMENT}
           \subsection{User Registration}
The system should allow administrators to register individuals by capturing and storing their facial features.
Registration should be straightforward and user-friendly.
\subsection{Face Detection}
Accurate face detection algorithms should be implemented to identify and locate faces in images or video streams.
\subsection{Face Recognition}
Employ robust facial recognition algorithms for accurate identification of registered individuals.
The system should provide a confidence score or percentage to indicate the level of certainty in the recognition process.
\subsection{Enrollment and Database Management}
The system should manage a secure database of enrolled users and their corresponding facial features.
Allow easy addition, modification, or removal of users from the database.

\subsection{Real-time Monitoring}
Provide real-time monitoring capabilities for administrators to view attendance status.
Generate alerts for late entries or other specified conditions.
 \section{NON-FUNCTIONAL REQUIREMENT}
These requirements are not needed by the system but are essential for the better
performance of sentiment engine. The points below focus on the non-functional
requirement of the system.
           \subsection{Performance}
\subsubsection{Response Time} The system should have low latency in recognizing faces and recording attendance.
\subsubsection{Throughput} It should be able to handle a specified number of facial recognition transactions per second.
\subsection{Scalabiity}
The system should be scalable to accommodate an increasing number of users and devices.
\subsection{Reliability}
The system should be highly reliable, minimizing the risk of false positives or negatives in facial recognition.
It should have mechanisms to handle system failures and recover gracefully.
\subsection{Availability}
The system should have a high level of availability to ensure that it is accessible when needed.
\subsection{Security}
\subsubsection{Data Security} Facial data and attendance records should be stored and transmitted securely, following industry best practices.
\subsubsection{Access Control}Implement robust access control mechanisms to restrict unauthorized access to the system.

        \section{FEASIBILITY STUDY}
        
A feasibility study for a Facial Attendance System is crucial to assess the viability of implementing such a system within an organization. The study typically examines various aspects to determine if the proposed solution is feasible from technical, economic, operational, and scheduling perspectives. Here are key components of a feasibility study for a Facial Attendance System:

\subsection{Technical Feasibility}

\subsubsection{Requirements Analysis} Assess the technical requirements of implementing a Facial Attendance System, including the necessary hardware, software, and networking components.
\subsubsection{System Compatibility} Evaluate the compatibility of the proposed system with existing infrastructure, databases, and devices.
\subsubsection{Facial Recognition Technology} Investigate the reliability and accuracy of available facial recognition algorithms and technologies.
\subsection{Economic Feasibility}

\subsubsection{Cost-Benefit Analysis} Evaluate the costs associated with developing, implementing, and maintaining the Facial Attendance System. Compare these costs with the expected benefits, including time savings, accuracy improvements, and potential cost reductions.
\subsubsection{Return on Investment (ROI)} Calculate the expected ROI over a specified period, considering both tangible and intangible benefits.
\subsection{Operational Feasibility}

\subsubsection{User Acceptance} Assess the willingness and ability of users (both administrators and employees) to adapt to and accept the new system.
\subsubsection{Training Requirements} Identify the training needs for users and administrators to effectively use and manage the Facial Attendance System.
\subsubsection{Integration with Existing Processes} Evaluate how seamlessly the system can integrate with existing attendance tracking and HR processes.